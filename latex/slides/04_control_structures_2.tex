%% Nothing to modify here.
%% make sure to include this before anything else

\documentclass[10pt,aspectratio=169]{beamer}
\usetheme{metropolis}
\usenavigationsymbolstemplate{}
\usepackage{tgcursor}

% packages
\usepackage{color}
\usepackage{listings}

% color definitions
\definecolor{mygreen}{rgb}{0,0.6,0}
\definecolor{mygray}{rgb}{0.5,0.5,0.5}
\definecolor{mymauve}{rgb}{0.58,0,0.82}

% re-format the title frame page
\makeatletter
\def\supertitle#1{\gdef\@supertitle{#1}}%
%\setbeamertemplate{title page}
%{
%  \vbox{}
%  \vfill
%  \begin{centering}
%  \begin{beamercolorbox}[sep=8pt,center]{title}
%      \usebeamerfont{supertitle}\@supertitle
%   \end{beamercolorbox}
%    \begin{beamercolorbox}[sep=8pt,center]{title}
%      \usebeamerfont{title}\inserttitle\par%
%      \ifx\insertsubtitle\@empty%
%      \else%
%        \vskip0.25em%
%        {\usebeamerfont{subtitle}\usebeamercolor[fg]{subtitle}\insertsubtitle\par}%
%      \fi%
%    \end{beamercolorbox}%
%    \vskip1em\par
%    \begin{beamercolorbox}[sep=8pt,center]{author}
%      \usebeamerfont{author}\insertauthor
%    \end{beamercolorbox}
%    \begin{beamercolorbox}[sep=8pt,center]{institute}
%      \usebeamerfont{institute}\insertinstitute
%    \end{beamercolorbox}
%    \begin{beamercolorbox}[sep=8pt,center]{date}
%      \usebeamerfont{date}\insertdate
%    \end{beamercolorbox}\vskip0.5em
%    {\usebeamercolor[fg]{titlegraphic}\inserttitlegraphic\par}
%  \end{centering}
%  \vfill
%}
\makeatother

% insert frame number
\expandafter\def\expandafter\insertshorttitle\expandafter{%
      \insertshorttitle\hfill%
\insertframenumber\,/\,\inserttotalframenumber}

% preset-listing options
\lstset{
  backgroundcolor=\color{white},
  % choose the background color;
  % you must add \usepackage{color} or \usepackage{xcolor}
  basicstyle=\footnotesize,
  % the size of the fonts that are used for the code
  breakatwhitespace=false,
  % sets if automatic breaks should only happen at whitespace
  breaklines=true,                 % sets automatic line breaking
  captionpos=b,                    % sets the caption-position to bottom
  commentstyle=\color{mygreen},    % comment style
  % deletekeywords={...},
  % if you want to delete keywords from the given language
  extendedchars=true,
  % lets you use non-ASCII characters;
  % for 8-bits encodings only, does not work with UTF-8
  frame=single,                    % adds a frame around the code
  keepspaces=true,
  % keeps spaces in text,
  % useful for keeping indentation of code
  % (possibly needs columns=flexible)
  keywordstyle=\color{blue},       % keyword style
  % morekeywords={*,...},
  % if you want to add more keywords to the set
  numbers=left,
  % where to put the line-numbers; possible values are (none, left, right)
  numbersep=5pt,
  % how far the line-numbers are from the code
  numberstyle=\tiny\color{mygray},
  % the style that is used for the line-numbers
  rulecolor=\color{black},
  % if not set, the frame-color may be changed on line-breaks
  % within not-black text (e.g. comments (green here))
  stepnumber=1,
  % the step between two line-numbers.
  % If it's 1, each line will be numbered
  stringstyle=\color{mymauve},     % string literal style
  tabsize=4,                       % sets default tabsize to 4 spaces
  title=\lstname
  % show the filename of files included with \lstinputlisting;
  % also try caption instead of title
}

% macro for code inclusion
\newcommand{\includecode}[2][c]{
	\lstinputlisting[caption=#2, style=custom#1]{#2}
}
	% nothing to do here
\usepackage[english]{babel}

\usepackage[utf8]{inputenc}

\newcommand{\course}{
	C introduction
}

\author{
}

\lstset{
	language = C,
	showspaces = false,
	showtabs = false,
	showstringspaces = false,
	escapechar = @,
	belowskip=-1.5em
}

\def\ContinueLineNumber{\lstset{firstnumber=last}}
\def\StartLineAt#1{\lstset{firstnumber=#1}}
\let\numberLineAt\StartLineAt

\makeatletter
%\def\beamer@verbatimreadframe{%
%	\begingroup%
%	\let\do\beamer@makeinnocent\dospecials%
%	\count@=127%
%	\@whilenum\count@<255 \do{%
%		\advance\count@ by 1%
%		\catcode\count@=11%
%	}%
%	\beamer@makeinnocent\^^L% and whatever other special cases
%	\beamer@makeinnocent\^^I% <-- PATCH: allows tabs to be written to temp file
%	\endlinechar`\^^M \catcode`\^^M=12%
%	\@ifnextchar\bgroup{\afterassignment\beamer@specialprocessframefirstline\let\beamer@temp=}{\beamer@processframefirstline}}%
%\makeatother
 % TODO modify this if you have not already done so

% meta-information
\newcommand{\topic}{
	Control structures 2
}
\usepackage{tikz}
\definecolor{orange}{RGB}{255,127,0}
\lstset{
  moredelim=**[is][\textit{ }]{§}{§},
}
% nothing to do here
\title{\topic}
\supertitle{\course}
\date{}

% the actual document
\begin{document}

\maketitle

\begin{frame}{Contents}
	\tableofcontents
\end{frame}

\section{Motivation}
\subsection{}
\begin{frame}
	\centerline{\includegraphics[scale=.32]{../img/friendship.jpg}}
	\bigskip
	Take a look at the right part. It is executed up to seven times.
\end{frame}

\section{Loops}
\subsection{}
\begin{frame}[fragile]{Loops}
	To repeat statements as long as a certain condition is met, C offers 3 different loops.
	\begin{lstlisting}[numbers=none,basicstyle=\itshape\footnotesize]
while (condition) {
	statement;
}
\end{lstlisting}
	\begin{lstlisting}[numbers=none,basicstyle=\itshape\footnotesize]
do {
	statement;
} while (condition);
\end{lstlisting}
	\begin{lstlisting}[numbers=none,basicstyle=\itshape\footnotesize]
for (initialization; condition; statement) {
	statement;
}
\end{lstlisting}
\end{frame}
\begin{frame}[fragile]{while}
	The execution of a loop is a continuous alternation between checking if the condition is still met and executing the statement(s).
	\begin{lstlisting}
int i = 2;
while (i > 0) {
	- -i;
}
printf("done\n");
\end{lstlisting}
	\begin{enumerate}[<+(1)->]
		\item Check (i $>$ 0) $\rightarrow$ \textbf{true} $\rightarrow$ go to line 3
		\item Decrement i $\rightarrow$ i now is \textbf{1}, go back to line 2
		\item Check (i $>$ 0) $\rightarrow$ \textbf{true} $\rightarrow$ go to line 3
		\item Decrement i $\rightarrow$ i now is \textbf{0}, go back to line 2
		\item Check (i $>$ 0) $\rightarrow$ \textbf{false} $\rightarrow$ go to line 4
		\item Print \textbf{done}
	\end{enumerate}
\end{frame}

\begin{frame}[fragile]{do...while}
	The difference between \textit{do...while} and \textit{while} is the order of executing the statement(s) and checking the condition.\\
	\bigskip
	The \textit{while} loop begins with checking, while the \textit{do...while} loop begins with executing the statement(s).
	\begin{lstlisting}[numbers=none]
int i = 3;
do {
	- -i;
} while (i < 1);
\end{lstlisting}
	\bigskip
	The Statement(s) in a \textit{do ... while} loop are executed at least once.
\end{frame}
\begin{frame}[fragile]{for}
	The For-Loop is comfortable for iterating. It takes three arguments.
	\begin{itemize}
		\item Initialization
		\item Condition
		\item Iteration statement
	\end{itemize}
	\bigskip
	For illustration, consider a program printing the numbers 1 to 10:
	\begin{lstlisting}[numbers=none]
int i;
for (i = 1; i <= 10; ++i) {
	printf("%d\n", i);
}
\end{lstlisting}
	\begin{itemize}
		\item \textit{i} is called an \textit{index} iterating from the given start to a given end value
		\item \textit{i, j, k} are commonly used identifiers for the index
	\end{itemize}
\end{frame}

\section{Miscellaneous}
\subsection{}
\begin{frame}[fragile]{Meanwhile...}
	Be careful, this
	\begin{lstlisting}[numbers=none]
while (1 > 0) {
	printf("Did you miss me?\n");
}
\end{lstlisting}
runs till the end of all days.\\
\ \\$\infty$ loops are common mistakes, and you will experience many of them.\\
Check for conditions that are always true.
\end{frame}

\begin{frame}[fragile]{for ever}
	The arguments for the \textit{for loop} are optional. E.g. if you already have defined your iterating variable:
	\begin{lstlisting}[numbers=none]
int i = 1;
for (; i <= 10; ++i)
	printf("%d\n", i);
\end{lstlisting}
	Or if you have the iteration statement in your loop body:
	\begin{lstlisting}[numbers=none]
for (i = 1; i <= 10;)
	printf("%d\n", ++i);	/* seems more like a while loop */
\end{lstlisting}
	And if you're not passing anything, it runs \textbf{for}ever:
	\begin{lstlisting}[numbers=none]
for (;;)
	printf("I'm still here\n");
\end{lstlisting}
Note: the semicolons are still there.
\end{frame}

\begin{frame}{Cancelling loops}
	\textit{break}
	\begin{itemize}
		\item Ends loop execution
		\item Moves forward to first statement after loop
	\end{itemize}\ \\\ \\
	\textit{continue}
	\begin{itemize}
		\item Ends current loop iteration
		\item Moves forward to next step of loop iteration
		\begin{itemize}
			\item\textit{while:} Jumps to condition
			\item\textit{for:} Jumps to iteration statement
		\end{itemize}
	\end{itemize}
\end{frame}

\begin{frame}[fragile]{Saving code lines}
	You can define variables inside the initialization part of a for loop.
	\begin{lstlisting}[numbers=none]
for (int i = 1; i <= 10; ++i) {
	printf("%d\n", i);
}
\end{lstlisting}
	\ \\\ \\In that case, the variable is only available inside the for loop (as if it was declared in the body).\\
	\begin{lstlisting}
int guess = 0;
int solution = 42;
for (int i = 1; guess != solution; ++i) {
	scanf("%d", &guess);
}
printf("Tries: %d\n", i);	/* invalid! */
\end{lstlisting}
	This feature was added in the C99 standard.
\end{frame}

\begin{frame}[fragile]{Compiler options}
	When calling \textit{gcc}, you can pass several options to it:\\
	\bigskip
	\begin{tabular}{|l|l|}
		\hline
		\textbf{option} & \textbf{description} \\\hline
		-std=c99 & Use C99 as the standard \\\hline
		-o $<$name$>$ & output file is \textit{name} instead of \textit{a.out} \\\hline
		-Wall & Enable all compiler warnings \\\hline
		-Wextra & Enable even more compiler warnings \\\hline
		-Werror & Treat warnings as errors \\\hline
	\end{tabular}\\
	\bigskip
	Example:
	\begin{lstlisting}[numbers=none]
$ gcc -std=c99 -o main main.c
\end{lstlisting}
\end{frame}

\begin{frame}[fragile]{A few words on style}
	\begin{itemize}
        \item Stay consistent after deciding whether to use or not to use braces on a single statement
		\item If you skip the loop body
		\begin{itemize}
			\item Leave a comment in your code
			\item Use an extra line for the empty statement
		\end{itemize}
	\end{itemize}
		\begin{lstlisting}[numbers=none]
for (i = 1; i < 9; printf("%d\n", ++i));	/* confusing */

for (i = 1; i < 9; printf("%d\n", ++i))		/* clear */
	;	/* do nothing */
\end{lstlisting}
\end{frame}

% nothing to do from here on
\end{document}
